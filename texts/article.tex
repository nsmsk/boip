
\documentclass[a4paper,12pt]{article}

\usepackage{ucs}
\usepackage[utf8x]{inputenc}
\usepackage[russian]{babel}
%\usepackage{cmlgc}
\usepackage{graphicx}
\usepackage{listings}
\usepackage{xcolor}
\usepackage{titlesec}
%\usepackage{courier}

\makeatletter
\renewcommand\@biblabel[1]{#1.}
\makeatother

\newcommand{\myrule}[1]{\rule{#1}{0.4pt}}
\newcommand{\sign}[2][~]{{\small\myrule{#2}\\[-0.7em]\makebox[#2]{\it #1}}}

% Поля
\usepackage[top=20mm, left=30mm, right=10mm, bottom=20mm, nohead]{geometry}
\usepackage{indentfirst}

% Межстрочный интервал
\renewcommand{\baselinestretch}{1.50}



% ------------------------------------------------------------------------------
% tcolorbox / tcblisting
% ------------------------------------------------------------------------------
\usepackage{xcolor}
\definecolor{codecolor}{HTML}{FFC300}

\usepackage{tcolorbox}
\tcbuselibrary{most,listingsutf8}

\tcbset{tcbox width=auto,left=1mm,top=1mm,bottom=1mm,
right=1mm,boxsep=1mm,middle=1pt}

\newtcblisting{myr}[1]{colback=codecolor!5,colframe=codecolor!80!black,listing only, 
minted options={numbers=left, style=tcblatex,fontsize=\tiny,breaklines,autogobble,linenos,numbersep=3mm},
left=5mm,enhanced,
title=#1, fonttitle=\bfseries,
listing engine=minted,minted language=r}

%%%%%%%%%%%%%%%%%%%%%%%%%%%%%%%%%%%%%%%

\begin{document}

%%%%%%%%%%%%%%%%%%%%%%%%%%%%%%%
%%%                         %%%
%%% Начало титульного листа %%%

\thispagestyle{empty}
\begin{center}


\renewcommand{\baselinestretch}{1}
{\large
{\sc Петрозаводский государственный университет\\
Институт математики и информационных технологий\\
Кафедра информатики и математического обеспечения
}
}

\end{center}


\begin{center}
%%%%%%%%%%%%%%%%%%%%%%%%%
%
% Раскомментируйте (уберите знак процента в начале строки)
% для одной из строк типа направления  - бакалавриат/
% магистратура и для одной из
% строк Вашего направление подготовки
%
 Направление подготовки бакалавриата \\
 01.03.02 --- Прикладная математика и информатика \\
% 09.03.02 --- Информационные системы и технологии \\
% 09.03.04 --- Программная инженерия \\
%%%%%%%%%%%%%%%%%%%%%%%%%
\end{center}

\vfill

\begin{center}
{\normalsize 
	Отчет по практике}

\medskip

%%% Название работы %%%
	{\Large \sc {Разработка приложения для тренировки устного счета }} \\
\end{center}

\medskip

\begin{flushright}
\parbox{11cm}{%
\renewcommand{\baselinestretch}{1.2}
\normalsize
	Выполнил:\\
% Выполнили:\\
%%% ФИО студента %%%
студента 1 курса группы 22103
\begin{flushright}
	Никита Сергеевич Самосюк \sign[подпись]{4cm}
\end{flushright}

%%% Второй участник %%%
% студента 1 курса группы 2210X
% \begin{flushright}
% 	И. О. Фамилия \sign[подпись]{4cm}
% \end{flushright}

%%%%%%%%%%%%%%%%%%%%%%%%%
% девушкам применять "Выполнила" и "студентка"
%%%%%%%%%%%%%%%%%%%%%%%%%
}
\end{flushright}

\vfill

\begin{center}
\large
    Петрозаводск --- 2020
\end{center}

%%% Конец титульного листа  %%%
%%%                         %%%
%%%%%%%%%%%%%%%%%%%%%%%%%%%%%%%

%%%%%%%%%%%%%%%%%%%%%%%%%%%%%%%%
%%%                          %%%
%%% Содержание               %%%

\newpage

\tableofcontents

%%% Содержание              %%%
%%%                         %%%
%%%%%%%%%%%%%%%%%%%%%%%%%%%%%%%

%%%%%%%%%%%%%%%%%%%%%%%%%%%%%%%%
%%%                          %%%
%%% Введение                 %%%

%%% В введении Вы должны описать предметную область, с которой связана %%%
%%% Ваша работа, показать её актуальность, вкратце определить цель     %%%
%%% разработки					       %%%


\newpage
\section*{Введение}
\addcontentsline{toc}{section}{Введение}

Цель проекта:

Разработка приложения с графическим интерфейсом пользователя, позволяющего
отрабатывать навыки быстрого счета в уме.

Задачи проекта: 
%%% Пример создания списков %%%
\begin{enumerate} 
\item	разработка игрового модуля, оценивающего количество арифметических выражений,
	вычисленных пользователем за 1 минуту;
\item	разработка игрового модуля, оцениващего способность пользователя вычислить
	выражение за 10 секунд;
\item	разработка модуля сохранения лучшего результата для каждого режима игры.
\end{enumerate}


%%% Пример добавления изображения %%%
%%% называйте изображение латиницей %%%

%%%                          %%%
%%%%%%%%%%%%%%%%%%%%%%%%%%%%%%%%

%%%%%%%%%%%%%%%%%%%%%%%%%%%%%%%
%%% Требования к приложению %%%
\newpage
\section{Требования к приложению}
% \subsection{Подраздел}

Следует заметить, что задача тренировки устного счета может сама по себе
иметь несколько разных формулировок. Можно, например, вычислить максимальное
число выражений за достаточное время, порядка одной минуты. В то же самое
время можно оценивать способность пользователя вычислить выражение за
короткое время - например, 10 секунд, то есть в более быстром темпе. Также
следует предусмотреть возможность регулировки сложности выражений. Таким образом,
при разработке приложения следует обеспечить возможность его работы в нескольких
игровых режимах с возможностью сохранения результата для каждого из них.

С учетом изложенного, к разрабатываемому приложению предъявляются следующие требования:

\begin{enumerate}
\item	приложение должно предлагать пользователю решать математические выражения
	в течении определенного времени;
\item	приложение должно предоставлять возможность выбора уровня сложности выражений;
\item	приложение должно предоставлять специальный режим, жестко ограничивающий
	время решения каждой задачи;
\item	приложение должно сохранять лучший результат для каждого режима игры;
\item	приложение должно иметь оконный графический интерфейс пользователя;
\item	приложение должно работать под управлением операционных систем Windows и Linux.
\end{enumerate}
 
%%%                                     %%%
%%%%%%%%%%%%%%%%%%%%%%%%%%%%%%%%%%%%%%%%%%%

%%%%%%%%%%%%%%%%%%%%%%%%%%%%%%%%%%%%%%%%%%%
%%%                                     %%%
%%% Проектирование приложения           %%%
\newpage
\section{Проектирование приложения}

Как отмечено ранее, приложение должно предоставлять пользователю возможность выбирать
между различными вариантами игры. Всего возможны 3 варианта:

\begin{enumerate}
\item	Легкий - игрок должен в течении минуты ввести правильные ответы на как можно
	большее число математических выражений, имеющих вид умножения двух двузначных
	чисел или сложения трех двузначных чисел. При вводе пользователем правильного
	ответа его счет увеличивается на единицу, и выражение заменяется другим. При
	вводе неправильного ответа предлагается попробовать еще раз.
\item	Средний - аналогичен легкому, но предлагаемые выражения более сложны, предлагается
	складывать и вычитать три трехзначных числа.
\item	Сложный - общее время игры неограничено, но на решение каждой задачи дается
	не более 10 секунд. Игра продолжается до первого неверного ответа или ответа,
	не введенного за 10 секунд.
\end{enumerate}

Приложение было спроектировано в объектно-ориентированном стиле и содержит следующие модули (классы):

\begin{enumerate}
\item класс основного окна приложения;
\item класс вкладки основного окна приложения, содержащий диалог выбора опций для начала игры, включая возможность ввода игроком своего имени и выбора режима игры
\item класс вкладки основного окна приложения, на которой собственно ведется игра, в которой пользователь должен за заданное время ввести ответы на как можно большее число арифметических задач;
\item класс вкладки основного окна приложения для специального игрового режима, являющийся потомком класса игровой вкладки и имеющий аналогичные интерфейсные функции, но отличающийся внутренней реализацией, ограничивая время решения каждой задачи, а не общее время игры;
\item класс вкладки основного окна приложения, содержащей список лучших результатов для каждого режима игры;
\item класс хранилища лучших результатов для каждого режима игры, отвечающий за хранение этих результатов в том числе между сеансами работы приложения.
\end{enumerate}

Основное окно приложения содержит функции переключения между различными вкладками в
зависимости от действий пользователя в текущей вкладке. При запуске программы отображается
вкладка с приглашением ввести имя пользователя и выбрать режим игры. Также на этой вкладке
находится кнопка просмотра лучших прошлых результатов для каждого игрового режима.

При нажатии на эту кнопку происходит переключение на вкладку результатов игры,
а при нажатии кнопки закрытия на ней - возврат к первой вкладке.

При нажатии на кнопку начала игры создается третья вкладка, содержащая математическое
выражение, поле для ввода его результата и поле с отображением таймера. На этой вкладке
происходит собственно игра. По окончании игры ее результат сохраняется в списке
лучших результатов, если он является лучшим, после чего активной становится вкладка
с лучшими результатами.

%%%                          %%%
%%%%%%%%%%%%%%%%%%%%%%%%%%%%%%%%

%%%%%%%%%%%%%%%%%%%%%%%%%%%%%%%%
%%%                          %%%
%%% Реализация приложения    %%%
\newpage
\section{Реализация приложения}

Приложение разработано на языке C++ с использованием библиотеки Qt для
создания кроссплатформенного графического интерфейса пользователя.

Основное окно приложения содержит в себе три вкладки, из которых в каждый
момент времени видна только одна. Две из этих вкладок существуют постоянно.
Первая из них содержит поле ввода имени пользователя, поле выбора режима игры
и кнопку начала игры. Вторая постоянно существующая вкладка содержит таблицу
с лучшим результатом для каждого режима игры. Одной из возможностей Qt является
возможность отслеживания элементом пользовательского интерфейса (называемом
виджетом в терминологии Qt) автоматически реагировать на каждое изменение
состояния другого объекта, в данном случае - объекта-хранилища лучших результатов.
Эта возможность позволяет виджету отображения рейтинга автоматически обновляться
при добавлении в хранилище новых результатов.

Третья вкладка содержит интерфейс самой игры - поле для вывода на экран
математического выражения и поля для ввода игроком результата этого
выражения. Эта вкладка создается динамически при каждом начале игры и
уничтожается в конце игры, так как ее поведение зависит от выбора режима
игры. Эта вкладка может быть объектом одного из двух классов - для
простого и среднего уровень сложности используется один класс виджета,
а для сложного уровня - другой класс виджета, производный от него, в котором
переопределена значительная часть поведения.

%%% Если необходимо вставками оформляются исключительно небольшие фрагменты кода.
%%% Для больших фрагментов используте приложение (пример после заключения)

Приведем для примера код объявления класса основного окна приложения.

\begin{verbatim}
/**
 * @brief Основное окно программы
 */
class GameWindow: public QWidget
{
    Q_OBJECT

public:
    /* Создать основное окно */
    GameWindow(QWidget *parent = nullptr);

public slots:
    /* Показать вкладку лучших результатов */
    void showRating();
    /* Скрыть вкладку лучших результатов */
    void hideRating();

    /* Начать игру с выбранным именем и режимом */
    void startGame(const QString & name, const QString & mode);
    /* Закончить игру */
    void stopGame();

private:
    /* Создать основной игровой виджет */
    PlayWidget * createPlayWidget(const QString & name, const QString & mode);

private:
    /* Набор виджетов, из которых видимым будет только один */
    QStackedWidget * mainStack;

    /* Приглашение ввести имя и выбрать режим игры */
    IntroWidget * introWidget;

    /* Вкладка с лучшими результатами по режимам игры */
    RatingWidget * ratingWidget;

    /* Вкладка, на которой проходит собственно игра */
    PlayWidget * playWidget;

    /* Хранилище лучших результатов */
    RatingObject * ratingObject;
};
\end{verbatim}

Обычно при использовании Qt основное окно приложения является объектом класса,
производного от QMainWindow, но для приложения, не имеющего меню, панели инструментов
и строки состояния можно использовать QWidget. Qt автоматически поместит виджет
не имеющий родительского, в отдельное окно. Макрос Q\_OBJECT объявляет внутри класса
служебные структуры Qt, позволяющие одним элементам управления реагировать на события
в других. Команда public slots является командой Qt, означающей, что перечисленные
после нее методы объекта могут быть вызваны через особый механизм связывания методов
с событиями, возникающими в других объектах. Благодаря этому механизму метод
showRating, делающий активной вкладку с лучшими результатами игроков, может быть
вызван вкладкой с приглашением ввести имя пользователя при нажатии на ней
кнопки просмотра рейтинга, без необходимости как-то учитывать наличие этой кнопки
в коде объекта осносного окна. Аналогично, метод hideRating, делающий основной вкладку
с приглашением ввести имя пользователя и начань игру, будет вызван при нажатии кнопки
<<OK>> на вкладке с лучшими результатами игры.

Объект QStackedWidget - это встроенный в Qt объект переключателя между вкладками.
Он определяет область в окне (в данном случае - все окно), в которой размещается
несколько элементов управления, из которых в каждый момент времени виден только один.
Объект ratingObject является хранилищем пользовательких результатов, включающим их
чтение с диска при создании объекта, запись на диск при его уничтожении и возможность
добавить новый результат с заменой существующего в случае, если новый лучше.

Больше примеров фрагментов исходного кода находится в приложениях A и Б. Вызов connect,
часто встречающийся в приложении А, является вызовом специального связывающего
механизма Qt, который позволяет при наступлении определенного события в одном
объекте (например, нажатии на кнопку или нажатии на Enter на поле ввода) вызвать
метод другого объекта (например, обработчика нажатия на кнопку в ее родительском
окне). Приложение Б содержит код создания основного игрового виджета. Здесь видно,
что в зависимости от выбранного пользователем режима игры создаются объекты разных
классов, что позволяет добавлять новые режимы игры в виде новых подклассов класса
PlayWidget, в том числе существенно отличающихся от родительского класса.

%%%                          %%%
%%%%%%%%%%%%%%%%%%%%%%%%%%%%%%%%

%%%%%%%%%%%%%%%%%%%%%%%%%%%%%%%%
%%%                          %%%
%%% Заключение               %%%

\newpage
\section*{Заключение}
\addcontentsline{toc}{section}{Заключение}

Было разработано приложение, предназначенное для тренироки устного счета в различных
режимах. Приложение может работать в трех разных режимах:

\begin{enumerate}
\item	простой режим, предлагающий решить максимальное число арифметических
	задач в течении минуты;
\item	средний режим, отличающийся более сложными выражениями по сравнению
	с простым;
\item	сложный режим, дающий пользователю не более 10 секунд на каждую задачу
	без права ошибиться.
\end{enumerate}

Приложение может сохранять лучший из достигнутых до сих пор результатов для
каждого режима игры. Программа имеет оконный графический интерфейс и работает в
операционных системах Windows и Linux.

Одним из возможных направлений дальнейшей доработки приложения является его
интернационализация. В данный момент интерфейс приложения реализован на английском
языке, что является хорошей практикой для приложений, использующих Qt. Встроенные
в эту библиотеку механизмы интернационализации позволяют легко добавлять в программу
переводы интерфейса на разные языка без необходимость модификации остального исходного
кода или сопровождения отдельной версии для каждого языка.
%%%%%%%%%%%%%%%%%%%%%%%%%%%%%%%%
%%%                          %%%

%%%%%%%%%%%%%%%%%%%%%%%%%%%%%%%%
%%%                          %%%
%%% Приложение               %%%

\newpage
\appendix
%\section*{Приложение}
%\addcontentsline{toc}{section}{Приложение}
%\titleformat{\section}[display]
%  {\normalfont\Large\bfseries}
%  {Приложение\ \thesection}
%  {0pt}{\Large\centering}
%\renewcommand{\thesection}{\Asbuk{section}}

\section*{Приложение А. Код инициализации основного окна программы}
\addcontentsline{toc}{section}{Приложение А. Код инициализации основного окна программы}

\begin{verbatim}
/**
 * @brief Инициализация основного окна программы
 * @param parent Родительский виджет, обычно nullptr
 */
GameWindow::GameWindow(QWidget *parent):
    QWidget(parent)
{
    /* Заголовок окна, минимальный размер */
    setWindowTitle(tr("Math game"));
    setMinimumSize(640, 480);

    /* Загружаем из файла конфигурации лучшие результаты */
    ratingObject = new RatingObject(this);

    /* Создаем вкладки для начала игры и лучших результатов */
    mainStack = new QStackedWidget(this);
    introWidget = new IntroWidget(this);
    ratingWidget = new RatingWidget(ratingObject, this);

    /* Реакция на нажатие кнопок начала игры, вызова лучших
     * результатов и возврата в меню
     */
    connect(
    	introWidget, SIGNAL(ratingButtonPressed()),
    	this, SLOT(showRating()));
    connect(
    	ratingWidget, SIGNAL(okPressed()),
    	this, SLOT(hideRating()));
    connect(
    	introWidget, SIGNAL(startButtonPressed(QString, QString)),
    	this, SLOT(startGame(QString, QString)));

    /* Добавляем вкладки начала игры и лучших результатов в
     * переключатель вкладок
     */
    mainStack->addWidget(introWidget);
    mainStack->addWidget(ratingWidget);
    mainStack->setCurrentWidget(introWidget);

    /* Разворачиваем текущую вкладку на все окно */
    QHBoxLayout * la = new QHBoxLayout();

    la->addWidget(mainStack);

    setLayout(la);
}
\end{verbatim}


\section*{Приложение Б. Код создания основного игрового виджета}
\addcontentsline{toc}{section}{Приложение Б. Код создания основного игрового виджета}

\begin{verbatim}
/**
 * @brief Создать основной игровой виджет
 * @param name Имя игрока
 * @param mode Режим игры
 * @return Виджет для игры. Его класс зависит от режима игры
 */
PlayWidget * GameWindow::createPlayWidget(const QString &name, const QString &mode)
{
    if (mode == tr("Hard")) {
        return new PlayWidgetHard(name, mode, this);
    } else {
        return new PlayWidget(name, mode, this);
    }
}
\end{verbatim}

%%% Ещё одно приложение
% \newpage
% \section*{Приложение Б.}
% \addcontentsline{toc}{section}{Приложение Б.}

%%%                          %%%
%%%%%%%%%%%%%%%%%%%%%%%%%%%%%%%%
\end{document}

