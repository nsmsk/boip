\documentclass[10pt]{beamer}
\usepackage[T1,T2A]{fontenc}
\usepackage[utf8]{inputenc}
\usepackage{hyperref}
\hypersetup{unicode=true}
\usepackage{fontawesome}
\usepackage{graphicx}
\usepackage[english,russian]{babel}

\usepackage[T1]{fontenc}
\usepackage{fontawesome}
\usepackage{PTSans} 
\mode<presentation>
{
  \usetheme[progressbar=foot,numbering=fraction,background=light]{metropolis} 
  \usecolortheme{default}
  \usefonttheme{default}
  \setbeamertemplate{navigation symbols}{}
  \setbeamertemplate{caption}[numbered]
} 

\let\textttorig\texttt
\renewcommand<>{\texttt}[1]{%
  \only#2{\textttorig{#1}}%
}

\usepackage{xcolor}
\definecolor{codecolor}{HTML}{FFC300}

\usepackage{tcolorbox}
\tcbuselibrary{most,listingsutf8}

\tcbset{tcbox width=auto,left=1mm,top=1mm,bottom=1mm,
right=1mm,boxsep=1mm,middle=1pt}

\newtcblisting{myr}[1]{colback=codecolor!5,colframe=codecolor!80!black,listing only, 
left=5mm,enhanced,
title=#1, fonttitle=\bfseries,
listing engine=minted,minted language=r}

\definecolor{mygreen}{HTML}{37980D}
\definecolor{myblue}{HTML}{0D089F}
\definecolor{myred}{HTML}{98290D}

\usepackage{listings}

\lstdefinelanguage{XML}
{
  morestring=[b]",
  morecomment=[s]{<!--}{-->},
  morestring=[s]{>}{<},
  morekeywords={ref,xmlns,version,type,canonicalRef,metr,real,target}
}

\lstdefinestyle{myxml}{
language=XML,
showspaces=false,
showtabs=false,
basicstyle=\ttfamily,
columns=fullflexible,
breaklines=true,
showstringspaces=false,
breakatwhitespace=true,
escapeinside={(*@}{@*)},
basicstyle=\color{mygreen}\ttfamily,
stringstyle=\color{myred},
commentstyle=\color{myblue}\upshape,
keywordstyle=\color{myblue}\bfseries,
}


% ------------------------------------------------------------------------------
% The Document
% ------------------------------------------------------------------------------
\title{Разработка приложения для тренировки счета в уме}
\subtitle{Отчет о проектной работе по курсу <<Основы информатики и программирования>>}
\author{Никита Сергеевич Самосюк}
\date{29 мая 2020}

\begin{document}

\maketitle

\begin{frame}[fragile]{Цель работы}
Разработка программы для тренировки счета в уме со следующими свойствами

\begin{enumerate}
\item	Наличие нескольких игровых режимов
\item	Сохранение лучших результатов для каждого из игровых режимов
\item	Графический интерфейс пользователя
\item	Работа под управлением ОС Windows и Linux
\end{enumerate}
\end{frame}

\begin{frame}[fragile]{Режимы работы программы}
Предусмотрено три игровых режима

\begin{enumerate}
\item	За минуту вычислить как можно больше простых арифметических выражений
\item	За минуту вычислить как можно больше сложных арифметических выражений
\item	Вычисление выражения за 10 секунд, игра до первой ошибки
\end{enumerate}
\end{frame}

\begin{frame}[fragile]{Проектирование программы}

Программа в объектно-ориентированном стиле, включает следующие классы

\begin{enumerate}
\item	Основное окно программы
\item	Подокно выбора режима игры и ввода имени пользователя
\item	Подокно для отображения лучших результатов по каждому режиму игры
\item	Объект для хранения результатов по каждому режиму игры
\item	Подокно для вывода выражений и запроса ответа пользователя
\item	Подокно для вывода выражений и запроса ответа пользователя в
	режиме 10 секунд на ответ
\end{enumerate}
\end{frame}

\begin{frame}[fragile]{Реализация программы}

\begin{enumerate}
\item	Реализация на языке C++
\item	Использование Qt для графического интерфейса пользователя
\end{enumerate}
\end{frame}

\begin{frame}[fragile]{Почему Qt}

\begin{enumerate}
\item	Библиотека для разработки графического интерфейса, работающего под
	всеми операционными системами;
\item	написана на C++, отсюда высокая производительность;
\item	включает библиотеку контейнеров, работающих быстрее стандартной библиотеки
	C++, благодаря неявному разделению данных (механизм копирования при
	записи);
\item	работа с данными в различных форматах, интернационализация и много
	других возможностей.
\end{enumerate}
\end{frame}

\begin{frame}[fragile]{Итоги}

\begin{enumerate}
\item	Реализовано кроссплатформенное графическое приложение для тренировки устного
	счета;
\item	три возможных режима работы программы, для каждого сохраняется лучший
	результат;
\item	возможные направления дальнейшей доработки:
	\begin{itemize}
	\item	поддержка разных языков интерфейса;
	\item	поддержка других режимов игры;
	\item	хранение истории результатов пользователя.
	\end{itemize}
\end{enumerate}

\end{frame}

\begin{frame}[standout]
    Спасибо за внимание ~\alert{!}~
\end{frame}

\end{document}

